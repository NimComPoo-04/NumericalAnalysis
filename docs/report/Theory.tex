\documentclass{article}

\usepackage{amsmath}
\usepackage{amssymb}
\usepackage{color}
\usepackage{hyperref}
\hypersetup{
	colorlinks=true,
	linktoc=all,
	linkcolor=black
}

\begin{document}

\tableofcontents
\hypersetup{linktocpage}
\newpage

\section*{Numerical Integration}

It is not always possible to compute integrals of functions analytically, so methods have been
developed to compute values of these integrals through iterative algorithmic processes. One such
integrals which requires assistance of a proof solver is the gamma function.
\begin{align*}
	&\Gamma(z) = \int^\infty_0 t^{z-1} e^{-t} dt &\forall z \in \mathbb{C}\\
	&\Gamma(x) = (x - 1)! &\forall x \in \mathbb{N}
\end{align*}

Gamma function is very difficult to solve analytically. The value of the gamma function for the real numbers
grows in the order of $O(n!)$, which makes it a perfect candidate for testing numerical integral software.
In this paper we try to develop the techniques required to compute gamma function and other integrals whose
values are hard to compute analitically.

\subsection*{Functions of multiple variables and their Integrals}

Functions which operate on multiple variables are generally hard to comprehend than simple scaler functions.
The domain of the function is defined to be on multiple cartesian products of the real numbers.

e.g,
\begin{align*}
	&f : \mathbb{R}^{n} \rightarrow \mathbb{R}^{n}\\
	&f(x, y) = \sin(x) + \sin(y)
\end{align*}

Integrating such a function first requires that we define the bounds in terms of an equation. This would allow
the limits of the inner integral to vary with the variable of the outer integral. The most common form of such
a region is a rectangle which requires no modification to the original integral of an 1-variable function. However
a circular region is also common defined with the equation.

e.g,
\begin{align*}
	&\iint\limits_{x^2 + y^2 \le r^2} F(x, y) dA\\
	=\quad\int^r_{-r}&\int^{\sqrt{r^2 - y^2}}_{-\sqrt{r^2 - y^2}} F(x, y) dx dy\\
\end{align*}

\subsection*{Numerical aproximation of integrals}

Out of all the methods available for numerical integration we choose simpsons and trapezoidal rule. This is
simply due to the fact that these methods are iterative in nature and provide good compromise between minimization
of execution time and error.

Let us consider a function,
\begin{equation*}
f : \mathbb{R} \rightarrow \mathbb{R}
\end{equation*}

Let us consider $n$ equispaced nodes such that,
\begin{align*}
f_i &= f(x_i) &\forall i \in \mathbb{W}, i < n\\
h &= \frac{x_n - x_0}{n}\\
p &= \frac{x - x_0}{h}\\
\end{align*}

Let $L$ be a function approximating function $f$,
\begin{align*}
\Delta f_i &= f_{i+1} - f_{i}\\
L(x) &= \sum_{i = 0}^n \left( \frac{\Delta^i f_0}{i!} \prod_{j = 0}^i (p - j)  \right)\\
\end{align*}

%Integrating $f$ by utilizing this aproximation yeilds,

Let,
\begin{align*}
a = max \left( x_i \right)\\
b = min \left( x_i \right)
\end{align*}

Now,
\begin{align*}
\int^a_b f(x) dx &= \int^a_b L(x) dx\\
&= \int^a_b \sum_{i = 0}^n \left( \frac{\Delta^i f_0}{i!} \prod_{j = 0}^i (p - j)  \right) dx\\
&= \sum_{i = 0}^n \left( \frac{\Delta^i f_0}{i!} \int^a_b \prod_{j = 0}^i (p - j) dx \right)\\
&= \sum_{i = 0}^n \left( \frac{\Delta^i f_0}{i!} \int^n_0 \prod_{j = 0}^i (p - j) dp \right)
\end{align*}

For trapezoidal rule we consider only 2 nodes,
\begin{align*}
&\sum_{i = 0}^2 \left( \frac{\Delta^i f_0}{i!} \int^2_0 \prod_{j = 0}^i (p - j) dp \right)\\
=& f_0 + (f_1 - f_0) \int^2_0 p dp\\
=& f_0 + (f_1 - f_0) \left[ \frac{p^2}{2} \right]_0^2\\
=& f_0 + (f_1 - f_0) \cdot 2\\
=& f_0 + 2 \cdot f_1 - 2 \cdot f_0)\\
=& 2 \cdot f_1 - f_0)\\
\end{align*}

\subsection*{Computing Improper Integrals}

\end{document}
